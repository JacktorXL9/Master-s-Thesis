The purpose of this work is to take the well known hyperfine splitting in the ground state of hydrogen, and investigate two-photon transitions from the lower hyperfine level to the upper hyperfine level to see if the resulting absorption coefficient is a significant correction to astrophysical and cosmological phenomena. We are interested to see if the absorption coefficient is significant enough that it will provide a significant correction to light that is observed from distant stars. As the light travels through space it is absorbed by the hydrogen atoms through which it passes. The relevant frequencies are microwave frequencies between 0 and the transition frequency of the hyperfine splitting in the ground state of hydrogen being $\nu =$ $1,420,405,751.800$ Hz. The 21 cm $\lambda=c/\nu$ wavelength for this transition wavelength has already made large contributions to our understanding of the universe, and we are interested to see if this particular transition can be a significant correction. In addition to this we will be investigating the role of this transition in Coherent Anti-Stokes Raman Scattering (CARS). 

The process of two-photon transitions was first proposed by Nobel Laureate Maria Goeppert-Mayer in her doctoral dissertation in 1931. It was first described as a transition where ``an excited atom can divide its excitation energy into two light quanta, the sum of their energies is equal to the excitation energy, but each can be an arbitrary value" \cite{maria}. She also theorized that the reverse process was also possible being ``the case that two light quanta, whose sum of frequencies is equal to the excitation frequency of the atom, work together to excite the atom" \cite{maria}. This concept was originally derived within the formulation of the Raman effect where a photon would interact with an atom and scatter with a different frequency. The main difference between these two processes is that two-photon absorption or decay deals with either a double annihilation or creation of two photons whereas the Raman effect deals with the annihilation then creation of photon or vice versa. 

Through advancements in laser technology we know that these transitions are indeed possible and have made large contributions to our understanding of the physical world. Quantum mechanically these transitions are dominant between states that would otherwise be forbidden. For example in hydrogen the 1s to 2s electric dipole transition is forbidden by parity and angular momentum selection rules, and thus cannot occur with a single photon process, two-photon processes are allowed for such a transition. Extensive work has been done on this process for hydrogen and helium-like atoms by Drake in ``\textit{Spontaneous two-photon decay rates in hydrogen-like and helium-like ions}" \cite{draketwopA}. In this paper the decay rate increases in proportion to $Z^6$, and so increases to \((2.993\pm 0.012) \cross 10^{10}\) s$^{-1}$ for Kr$^{34+}$

%the non-relativistic two photon decay rate is calculated for all ions up to \(Z=36\), finishing on \(Kr^{+34}\) with a decay rate of \((2.993\pm 0.012) \cross 10^{10}s^{-1}\)

Relativistic effects have also been calculated for this transition by Goldman and Drake in their paper ``\textit{Relativistic two-photon decay rates of 2 \(s_{1/2}\) hydrogenic ions}" \cite{draketwoprel}. In their work all relativistic effects, retardation effects and all combinations of photon multipoles are taken into account. These effects are included because they make a significant contribution for all ions up to \(Z=100\)


In addition to this, extensive work has been done on the two-photon decay of the singlet and triplet metastable states of helium-like ions \cite{dalgarnohelium}. In this paper the decay rates are calculated for the heliumlike ions up to neon. This work is of great relevance to the present investigation. 
%It highlights the particular difference in two-photon transitions between exchanges of different level of angular momenta particularly the difference between an exchange of 2 units or 0 units of angular momenta and 1 unit of angular momenta in a two-photon process. The more common process is the exchange of 2 or one angular momenta, as this is a two photon process, each photon carries with it one unit of angular momenta and in this case they either add together to give 2 units of momenta or 0 units. This can be seen in the paper for the two-photon decay of the singlet metastable state of the helium like ion. 
%\begin{align}
%\begin{split}
%    A(v_1)dv_1=(1024\pi^6 e^4/h^2c^6)
%    \abs{(\Vec{u}_1\cdot \Vec{u}_2)^2}_av
%    v_1^3 v_2^3\\
%    \abs{\sum_{n'=2}^{\infty} 
%    \bigg(\frac{\bra{1 ^1S}P_z
%    \ket{n'^1P}\bra{n'^1P}
%    P_z
%    \ket{1^1S}}{v(2^1S-n'^1P)+v_2}+\frac{\bra{1 ^1S}P_z
%    \ket{n'^1P}\bra{n'^1P}
%%    P_z
%    \ket{1^1S}}{v(2^1S-n'^1P)+v_1}\bigg)}^2 dv_1
%    \end{split}
%\end{align}

%Where the average polarizabilities of the two photons is given as a dot product.
%\begin{equation}
%    \abs{(\Vec{u}_1\cdot \Vec{u}_2)^2}_{AV}
%\end{equation}
%In addition to this dot product the exchange of two or zero units of angular momenta is represented by a sum of the two scenarios of the photons exchanged. These two scenarios representing the order of the photons as they enter the system. This is due to the fact that the angular momentum creates the dot product and the equation.
%\begin{equation}
%    \Vec{u}_1\cdot \Vec{u}_2=\Vec{u}_2\cdot \Vec{u}_1
%\end{equation}

%In contrast when a transition requires two photons to exchange only a single unit of angular momenta they must approach the system at different angles. This is shown by the two-photon decay of the triplet metastable state of the helium like ion. 
%\begin{align}
%    \begin{split}
%        A(v_1)dv_1=(1024\pi^6 e^4/3h^2c^6)
%    \abs{(\Vec{u}_1\cross \Vec{u}_2)^2}_av
%    v_1^3 v_2^3\\
%    \bigg|\sum_{n',n''}^{\infty} 
%    (2 ^3S ||P||n'^3P)
%    \epsilon_{n'n''}
%    (n''^1P||P||1^1S)\\
%%    \cross \bigg(
%    \frac{1}{v(2^3S-n'^3P)+v_2}
%    -\frac{1}{v(2^3S-n''^1P)+v_2}\\
%    -\frac{1}{v(2^3S-n'^3P)+v_1}
%    +\frac{1}{v(2^3S-n''^1P)+v_1}
%    \bigg)
%    \bigg|^2 dv_1
%    \end{split}
%\end{align}
%In the formula for the decay rate the sum over all polarizations of the two photons is shown as an average of a cross product.
%\begin{equation}
%    \abs{(\Vec{u}_1\cross \Vec{u}_2)^2}_{AV}
%\end{equation}
%When this cross product is taken away from the bra-kets it leaves behind a minus sign as opposed to a plus sign seen in singlet decay. This is due to the fact that
%\begin{equation}
%    \Vec{u}_1\cross \Vec{u}_2=-\Vec{u}_2\cross \Vec{u}_1
%\end{equation}
%This transition is very similar to the transition being studied as it also is an exchange of 1 angular momenta in a two-photon process. In addition to this is a mixing of the energy states between the two fine structure levels of the intermediate p states of triplet and singlet p.
%\begin{equation}
%\begin{split}
%    \frac{1}{v(2^3S-n'^3P)+v_2}
%    -\frac{1}{v(2^3S-n''^1P)+v_2}\\
%    -\frac{1}{v(2^3S-n'^3P)+v_1}
%    +\frac{1}{v(2^3S-n''^1P)+v_1}
%\end{split}
%\end{equation}
%This will also occur in the studied transition.

A group at John Hopkins University\cite{recombinationB} has used two photon transitions and more specifically the two photon decay rate of the 2s state of hydrogen as a correction to recombination so that we can have a more complete picture of our universe closer to the Big Bang. Other two-photon transitions are also considered for this purpose such as higher energy s-states and higher energy d-states  \cite{recombinationA}.

In addition to the work that has been done in two-photon processes, extensive work has been done on the topic of hyperfine splitting. In early astronomy, astronomers detected a very distinct hum coming from the center of our galaxy. The now famous 21 cm line in astrophysics was originally theorized by H. C. Van de Hulst in 1945. He was assigned as a student to find what spectral line could exist at radio frequency. He started with hydrogen naturally as it is the most abundant element in the universe, and that is as far as he needed to go. He found that the hyperfine levels of ground state hydrogen would produce radio waves of this frequency. With this new information astronomers were able to detect radio waves of this frequency and use it to find hydrogen throughout our galaxy. This all culminated in a paper published in 1954 by Van der Hulst and two of his colleagues where they proposed that our galaxy had a spiral structure \cite{astro}.

Since the hyperfine splitting scales as \(Z^3\) with $Z$ being the nuclear charge, there are several applications for hyperfine structure transitions in heavier ions. For sufficiently large $Z$ the hyperfine splitting will actually overtake the fine structure splitting of the ion \cite{woodgate}. 

Very recently the hyperfine splitting in the ground state of the heavy metal ion for Bismuth, Bi\(^{+82}\) has been studied extensively \cite{mangeticmome}. The experiment revealed that there was a strong disagreement between the theoretical magnetic moment of bismuth and the experimentally derived value. This magnetic moment was then determined experimentally again using the comparison between the energies of the hyperfine transitions in ground state hydrogenlike Bi\(^{+82}\) and lithiumlike Bi\(^{+80}\). This seemed to have solved the ``hyperfine puzzle" and the theoretical magnetic moment was recalculated and found to be in closer agreement with experiment \cite{hyperfinepuzz}.

Two-photon processes are not simply limited to absorption and decay. In fact they are the basis for Raman and Rayleigh scattering processes. A photon is absorbed and then a separate one is emitted and vice versa thus making a two-photon transition. Work has been done by Dalgarno and Sadeghpour in finding the Raman and Rayleigh scattering cross sections in both hydrogen and caesium \cite{dalgarnoraman}. They used an inhomogeneous differential equation method to calculate this cross section. We will be looking into their work for our own work on calculating the Raman Scattering cross section for the hyperfine levels of hydrogen. 

Throughout this work we will detail all the factors that go into our calculations. In the next chapter we will detail the structure of a hydrogen atom and show how the coupling of the angular momentum leads to energy splitting within the principal quantum states. In the third chapter we will discuss how we use a discrete variational basis set to represent the entire spectrum of hydrogen. The fourth chapter will contain a brief summary of the physics that is involved in a two photon process. This chapter will also contain the results for the decay rate, absorption coefficient, and Raman scattering cross sections for hydrogen and all stable spin 1/2 hydrogenlike isotopes. Conclusions and future work will be presented within the final chapter.


















