\thispagestyle{plain}
%\begin{center}
%%    \Large
%    \textbf{Abstract}
%\end{center}
% We have shown that the the transition between the hyperfine energy levels of hydrogen can be connected by means of a two-photon electric dipole process. From this we have calculated the two-photon decay rate of the hyperfine levels in the ground state of hydrogen and shown that it does not need to be considered as a significant correction to recombination. From this we have also calculated the Einstein B coefficient for the absorption rate and applied to to a few scenarios to see if it would make a significant impact to astrophysics. We have also taken these energy levels and treated them as a Raman scattering process and calculated the relevant cross section.

%We have extended these calculations to other hydrogenic ions. We have made an approximation to the hyperfine splitting in the ground state of heavier ions and used this to calculate the decay rate, absorption coefficient and Raman scattering cross section for all stable nuclear spin $1/2$ isotopes from helium to bismuth.

The hyperfine levels for the ground state of hydrogen are normally connected by magnetic dipole transitions.  However, for astrophysical applications, it may be that two-photon electric dipole processes are also important over cosmological distances. This thesis sets up a general mathematical formalism for the calculation of two-photon processes within the hyperfine structure of an atom with spin $1/2$, and applies it to the ground state of hydrogen. Rates are calculated for both spontaneous emission and absorption of radiation. Due to an extraordinary degree of cancellation in the matrix elements, the rates turn out to be too small to have a significant impact on astrophysical processes involving the absorption and emission of radiation. The results are nevertheless important in reducing the uncertainty due to possible contributions from these processes.  Other two-photon processes involving Raman scattering are also considered and the relevant cross sections calculated. The calculations of the decay rate, absorption coefficient and Raman scattering cross section are extended to heavier hydrogenic ions as a function of the nuclear charge $Z$, using an approximate interpolation formula for the hyperfine splitting in the ground state. Results are presented for the hydrogenlike ions from He$^+$ to Pb$^{81+}$. Modifications for the case of isotopes with a nuclear spin greater than $1/2$ are briefly considered. 