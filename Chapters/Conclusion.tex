This chapter will detail the future potential work for this project as well as some conclusions that can be gathered from the data. Starting off with our two-photon decay rate for the hyperfine transition in ground state hydrogen, we can see that our value leads to a lifetime of $10^{61}$ years. This is one of the longest lifetimes ever calculated in all of atomic physics. The significance is that it removes the need to include the decay rate as a source of uncertainty in cosmological applications, and hence it reduces the uncertainty. If we look back at how the single-photon decay of the excited hyperfine 1s state in hydrogen was used in astrophysics, we can see that even though the lifetime was several million years it was still significant over a galactic time scale. When considering the fact that the two photons can be of any frequency in the hyperfine range, if someone were to detect a photon from this process, they would have no idea if it were red shifted or blue shifted. 

We can see from the application of the absorption coefficient to the cosmic microwave background radiation that the gain rate is still extremely small. Even when considering cosmological distances this value would not lead to a great deal of lost photons. The Raman scattering coefficients are equally as small. When considering a cross section in yoctobarns, this would be extremely difficult to reproduce in the laboratory.

We also calculated these three values for all stable spin $1/2$ hydrogenlike isotopes up to bismuth. We scaled our calculations with the approximation of the hyperfine splitting in the ground state as well as the scaling to energy and distance relative to the Z value for the isotope. The decay rate was the most dramatically affected, increasing by almost 30 orders of magnitude relative to the hydrogen case. This is due to the fact that the decay rate has a rough scaling with the hyperfine splitting of $\omega^6$ so it was most affected. The absorption coefficient however has very little dependence on the hyperfine splitting and therefore only increased by a few orders of magnitude as it was scaled up with $Z$. The Raman scattering cross section scales with the energy of the absorbed and emitted photon. However these frequencies are dependent on the energy difference between the ground state and the first intermediate state. This caused the Raman scattering cross section to not scale as dramatically as the decay rate. although it increases by several orders of magnitude as it scales with $Z$, it never leaves the range of yoctobarns and is therefore likely not observable in the laboratory.

However there can likely be still more results that can be derived from this particular project. There are several isotopes that were not considered in this project. It would be beneficial to see if the angular momentum algebra for the isotopes without nuclear spin $1/2$ yield the same dependence on the fine structure energy splitting of the intermediate p-states. In our work we found that when summing over the values of $J$ for the intermediate p-states we arrived at a minus sign separating the two fine structure levels. If we did not consider the energy splitting between these two levels then the equation would become zero. It would be of interest to see if the same minus sign appears for a nuclear spin $3/2$ particle. If this is the case then all of the values calculated in this work would no longer depend on the fine structure energy of the intermediate p-states. For the case of hydrogen for the first excited state we have
\begin{equation}
    \omega_{p\frac{3}{2}-p\frac{1}{2}}=\frac{\alpha^2Z^4}{4n^3}=1.665\cross10^{-6}
\end{equation}
which squared leads to an overall multiplying factor of $10^{-12}$ to the first set of transitions integrals. If this is removed then we would increase the results by 12 orders of magnitude and would also open up a wide variety of hydrogenic isotopes to look at. \cite{demidov1962two}